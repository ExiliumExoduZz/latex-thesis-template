%!TEX root = ../thesis.tex

%%%%%%%%%%%%%%%%%%%%
% Konfiguration in Anlehnung an https://github.com/FHDWPB/latex-thesis-template
% bestehend aus: Mark Müller, Willi Nüßer, Waldemar Penner, Ulrich Reus Frank Plass, Oliver Tribeß, Daniel Hintze
% 
% Einige Settings wurden angepasst, sodass das Template den Anforderungen am Standort Bergisch Gladbach entspricht.
%%%%%%%%%%%%%%%%%%%%


%%%%%%%%%%%%%%%%%%%%
% Generelle Konfiguration
%%%%%%%%%%%%%%%%%%%%

% Setzen der Dokumentenclass
\documentclass[%
paper=A4,         % alle weiteren Papierformat einstellbar
fontsize=12pt,    % Schriftgröße (12pt, 11pt (Standard))
BCOR=12mm,         % Bindekorrektur, bspw. 1 cm
DIV=14,            % breiter Satzspiegel
parskip=half*,    % Absatzformatierung s. scrguide 3.1
headsepline,      % Trennline zum Seitenkopf
footsepline,     % Trennline zum Seitenfuß
%normalheadings,  % Überschriften etwas kleiner (smallheadings)
listof=totoc,     % Tabellen & Abbildungsverzeichnis ins Inhaltsverzeichnis
%bibtotoc,        % Literaturverzeichnis im Inhalt
%draft            % Überlangen Zeilen in Ausgabe gekennzeichnet
footinclude=false,% Fußzeile in die Satzspiegelberechnung einbeziehen
headinclude=true, % Kopfzeile in die Satzspiegelberechnung einbeziehen
toc=sectionentrywithdots,
final             % draft beschleunigt die Kompilierung
]
{scrartcl}

% Setzen der Seitenränder
\usepackage[left=40mm, right=20mm, top=25mm, bottom=25mm]{geometry}

% Setzen der Ebenentiefe der Nummerierung
\setcounter{secnumdepth}{3}

% Setzen der Gliederungstiefe im Inhaltsverzeichnis
\setcounter{tocdepth}{3}

% Für die Definition eigener Kopf- und Fußzeilen
\usepackage{fancyhdr}

% Für die Verwendung von Grafiken
\usepackage{graphicx}

% Für Tikz Grafiken
\usepackage{tikz}

% Neue Deutsche Rechtschreibung und Deutsche Standardtexte
\usepackage[ngerman]{babel}


%%%%%%%%%%%%%%%%%%%%
% Schriftart und Überschriften
%%%%%%%%%%%%%%%%%%%%

% Times New Roman as default Schriftart
\usepackage{fontspec}
\setmainfont{Times New Roman}
% Überschreiben der Sections von scrarticel, sodass auch die Überschriften Times new Roman sind!
\newfontfamily\sectionfont{Times New Roman}
\addtokomafont{section}{\sectionfont}
\addtokomafont{subsection}{\sectionfont}
\addtokomafont{subsubsection}{\sectionfont}
% Change Schriftart im ToC (Inhaltsverzeichnis)
\addtokomafont{sectionentry}{\sectionfont}

% 1/2-zeiliger Zeilenabstand
\usepackage[onehalfspacing]{setspace}


%%%%%%%%%%%%%%%%%%%%
% Utilities und Fixes
%%%%%%%%%%%%%%%%%%%%

% Erlaubt die Benutzung von Farben
\usepackage{color}

% Ausgabe der aktuellen Uhrzeit für die Draft-Versionen
\usepackage{datetime}

\usepackage{scrhack}

%%%%%%%%%%%%%%%%%%%%
% Kopf und Fußzeilen
%%%%%%%%%%%%%%%%%%%%

% Style für Kopf- und Fußzeilenfelder
\pagestyle{fancy}
\fancyhf{}
\fancyhead[L]{\leftmark}
\fancyhead[R]{\thepage}
\renewcommand{\sectionmark}[1]{\markboth{#1}{#1}}
\fancypagestyle{plain}{}


%%%%%%%%%%%%%%%%%%%%
% Links
%%%%%%%%%%%%%%%%%%%%

\usepackage{hyperref}


%%%%%%%%%%%%%%%%%%%%
% Abkürzungsverzeichnis
%%%%%%%%%%%%%%%%%%%%

% Abkürzungsverzeichnis
\usepackage[acronym,         % create list of acronyms
    nonumberlist,
    toc,
    section,
    nopostdot,  % avoid dot after acronym
    hyperfirst=false,% don't hyperlink first use
%sanitize=none    % switch off sanitization as description % Deprecated
]{glossaries}
\newglossarystyle{mylist}{%
    \setglossarystyle{long}% base this style on the list style
    \renewcommand*{\glossaryentryfield}[5]{%
        \glsentryitem{##1}\textbf{##2} & ##3 \\}%
}
% Import der Abkürzungen
%Acronyms
\newacronym{iac}{IaC}{Infrastructure as Code}

%Glossary
\newglossaryentry{Glossar}
{
    name=Glossar,
    description={Ein Glossar ist eine alphabetisch geordnete Liste von Begriffen aus einem bestimmten Wissensgebiet mit den dazugehörigen Definitionen.}
}
\makeglossaries\makeglossaries


%%%%%%%%%%%%%%%%%%%%
% Literatur/Quellenverzeichnis
%%%%%%%%%%%%%%%%%%%%

%Literaturverzeichnis/Quellenverzeichnis
\usepackage[
    bibstyle=authoryear,
    citestyle=authoryear-fhdw,
    giveninits=true, %True = Vornamen werden abgekürzt
    %natbib=true,
    urldate=long,
    maxcitenames=2,
    maxbibnames=2,
    backend=biber]{biblatex}
\usepackage[german=quotes]{csquotes}
\bibliography{Quellen}
%Hiermit werden alle Quellen ins Verzeichnis aufgenommen auch ohne , dass sie Zitiert/genutzt werden.
\nocite{*}


%%%%%%%%%%%%%%%%%%%%
% Indexverzeichnis
% https://tug.org/tutorials/tugindia/chap16-scr.pdf Seite 9 ff.
%%%%%%%%%%%%%%%%%%%%

\usepackage{imakeidx}
\makeindex

%%%%%%%%%%%%%%%%%%%%
% Utilities und Fixes
%%%%%%%%%%%%%%%%%%%%
% Bessere URLs
\usepackage{url}

% Wilde Referenzieren auf Bilder, Tabellen, Seiten
\usepackage[german,capitalise]{cleveref}



%%%%%%%%%%%%%%%%%%%%
% Anhangsverzeichnis
% Copyright 2022 Mark Müller, Willi Nüßer, Waldemar Penner, Ulrich Reus Frank Plass, Oliver Tribeß, Daniel Hintze
% https://github.com/FHDWPB/latex-thesis-template
%%%%%%%%%%%%%%%%%%%%
% Anhangsverzeichnis: Credits: FHDWPB @Github https://raw.githubusercontent.com/FHDWPB/latex-thesis-template/master/config/Config.tex
\usepackage[nohints]{minitoc}

\makeatletter
\newcounter{fktnr}\setcounter{fktnr}{0}
\newcounter{subfktnr}[fktnr]\setcounter{subfktnr}{0}

\renewcommand\thesubfktnr{\arabic{fktnr}.\arabic{subfktnr}} %\thesubfktnr wird zu "Große_Nummer.Kleine_Nummer"
\newcounter{anhangcounter}
\newcommand{\blatt}{\stepcounter{anhangcounter}}

%Anhang Command
\newcommand{\anhang}[1]{\setcounter{anhangcounter}{0}\refstepcounter{fktnr}
\addcontentsline{fk}{subsection}{{\hspace*{0em}}Anhang~\thefktnr: #1}
\subsection*{{Anhang~\thefktnr: #1 }}
}
%Subanhang Command
\newcommand{\subanhang}[1]{\setcounter{anhangcounter}{0}\refstepcounter{subfktnr}
\addcontentsline{fk}{subsubsection}{{\hspace*{0em}}Anhang~\thesubfktnr: #1}
\subsubsection*{{Anhang~\thesubfktnr: #1 }}
}
%Anhangsverzeichnis Command
\newcommand{\anhangsverzeichnis}{
    \mtcaddsection{\subsection*{Anhangsverzeichnis \@mkboth{FKT}{FKT}}}
    \@starttoc{fk}
    \newpage
}
