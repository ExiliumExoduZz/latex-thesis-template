%!TEX root = ../thesis.tex

\section{Einleitung}
\label{sec:einleitung}

Hier ist eine kurze Einleitung für ein paar Befehle. Sonst ist nur quatsch hier.

First use \gls{iac}
Second use \gls{iac}

Glossary use \gls{Glossar}


Hier ist eine Beispielabbildung
\begin{figure}[hbt]
    \begin{minipage}{1\textwidth}

    \centering
    \includegraphics[width=0.5\textwidth]{abbildungen/fhdw}
    \caption{FHDW Logo}
    \label{fig:fhdw-logo}
    %\source{Eigene Darstellung}
    \end{minipage}
\end{figure}

BTW ist hier ein Verweis auf die Abbildung lel. In \cref{fig:fhdw-logo} siehst du das Logo der FHDW.

Und weil es so wild ist gibt es sogar noch eine Beispielfußnote \footnote{Okok}.
Hier sogar eine zweite\footnote{Vgl. OK, S. 23} weil die so krass sind.

Hier noch zitat \cite[Vgl.][]{ozel_infrastructure_2020}

Und noch eins kappa \footnote{\cite[13\psq]{rahman_as_2020}}

Ein vergleichendes Zitat.\footnote{\cite[vgl.][5\psqq]{rahman_as_2020}}

Und noch eins mit ff. \footnote{\cite[23\psqq]{rahman_as_2020}}

Hier setzen wir noch einen Index auf Infrastructure-as-Code \index{IaC}


\subsection{Unterkapitel - Der Wahnsinn}

Wow hier ist Platz.
